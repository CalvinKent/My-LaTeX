% Prepared by Calvin Kent
%
% Assignment Template v19.02
%
%%% 20xx0x/MATHxxx/Crowdmark/Ax
%
\documentclass[11pt]{article} %
\usepackage{CKpreamble}
\usepackage{CKassignment}
%
\begin{document}
	\pagenumbering{arabic}
	\renewcommand*{\qpoints}{10} %set default points to 10
	% Start of class settings ...
	\renewcommand*{\coursecode}{MATH xxx} % renew course code
	\renewcommand*{\assgnnumber}{Assignment \#00} % renew assignment number
	\renewcommand*{\submdate}{00 MONTH 2019} % renew the date
	\renewcommand*{\studentfname}{FName} % Student first name
	\renewcommand*{\studentlname}{LName} % Student last name
	\renewcommand*{\studentnum}{SNumber} % Student number
	\setfigpath
	% End of class settings	
	\pagestyle{crowdmark}
	\newgeometry{left=18mm, right=18mm, top=22mm, bottom=22mm} % page is set to default values
	\fancyhfoffset[L,O]{0pt} % header orientation fixed
	% End of class settings
	%%% Note to user:
	% CTRL + F <CHANGE ME:> (without the angular brackets) in CKpreamble to specify graphics paths accordingly.
	% The command \circled[]{} accepts one optional and one mandatory argument.
	% Optional argument is for the size of the circle and mandatory argument is for its contents.
	% \circled{A} produces circled A, with size drawn for letter A. \circled[TT]{A} produces circled A with size drawn for TT.
	% https://github.com/CalvinKent/My-LaTeX
	%%%
	% Crowdmark assignment start
\begin{qstn}[1] % qnumber, qname, qpoints
	Sample question
\begin{assgnenum}
	\qitem Sample part
	\begin{soln}
		content...
	\end{soln}
	\qitem Sample part
	\begin{soln}
		content...
	\end{soln}
	\begin{assgnenum}
		\qitem Sample subpart
		\begin{soln}
			content...
		\end{soln}
		\qitem Sample subpart
		\begin{soln}
			content...
		\end{soln}
	\end{assgnenum}
		\qitem Sample part
	\begin{soln}
		content...
	\end{soln}
\end{assgnenum}
\end{qstn}

\begin{qstn}[2]
	Sample question
\begin{soln}
	content...
\end{soln}
\end{qstn}

\begin{qstn}[3]
	Sample question
\begin{soln}
	content...
\end{soln}
\end{qstn}

\begin{qstn}[4]
	Sample question
\begin{soln}
	content...
\end{soln}
\end{qstn}

\begin{qstn}[5]
	Sample question
\begin{assgnenum}
	\qitem Sample part
	\begin{soln}
		content...
	\end{soln}
	\qitem Sample part
	\begin{soln}
		content...
	\end{soln}
	\begin{assgnenum}
		\qitem Sample subpart
		\begin{soln}
			content...
		\end{soln}
		\qitem Sample subpart
		\begin{soln}
			content...
		\end{soln}
		\begin{assgnenum}
			\qitem Sample subpart
			\begin{soln}
				content...
			\end{soln}
			\qitem Sample subpart
			\begin{soln}
				content...
			\end{soln}
		\end{assgnenum}
		\qitem Sample subpart
		\begin{soln}
			content...
		\end{soln}
	\end{assgnenum}
	\qitem Sample part
	\begin{soln}
		content...
	\end{soln}
\end{assgnenum}
\end{qstn}

\begin{qstn}[6]
	Sample question
\begin{soln}
	content...
\end{soln}
\end{qstn}

\begin{qstn}[15][(Bonus) ][49.5]
	This is a sample question demonstrating the abilities of this template. Source code for .sty and .tex files are available on Github:\\[\baselineskip]
	Templates: \url{http://www.student.math.uwaterloo.ca/~c2kent/LectureNotes/index.html}\\
	Github: \url{https://github.com/CalvinKent/My-LaTeX}
	\begin{itemize}
		\item Question number and points are customizable.
		\begin{itemize}
			\item It is possible to add question name (eg. (Bonus))
		\end{itemize}
		\item Page numbers and question number on the right change automatically.
		\item Question and parts are added to bookmarks automatically.
	\end{itemize}
	\begin{soln}
	\end{soln}
	\begin{assgnenum}
		\qitem Sample part
		\begin{soln}
			\begin{figure}[H]
				\centering
				\includegraphics{sample}
				\caption{Figures are numbered under question numbers.}
			\end{figure}
			\vfill
		\end{soln}
		\newpage
		\qitem Sample part
		\begin{soln}
			content...
			\vfill
		\end{soln}
		\begin{assgnenum}
			\qitem Sample part
			\begin{soln}
				content...
				\vfill
			\end{soln}
			\qitem Sample part
			\begin{soln}
				content...
				\vfill
			\end{soln}
			\newpage
			\qitem Sample part
			\begin{soln}
				content...
				\vfill
			\end{soln}
		\end{assgnenum}
		\qitem Sample part
		\begin{soln}
			content...
			\vfill
		\end{soln}
	\end{assgnenum}
\end{qstn}
 
\begin{qstn}[8]
	Sample question
\begin{soln}
	content...
\end{soln}
\end{qstn}

\begin{qstn}[9]
	Sample question
\begin{soln}
	content...
\end{soln}
\end{qstn}

\begin{qstn}[10]
	Sample question
\begin{assgnenum}
	\qitem Sample part
	\begin{soln}
		content...
	\end{soln}
	\qitem Sample part
	\begin{soln}
		content...
	\end{soln}
	\begin{assgnenum}
		\qitem Sample subpart
		\begin{soln}
			content...
		\end{soln}
		\qitem Sample subpart
		\begin{soln}
			content...
		\end{soln}
	\end{assgnenum}
	\qitem Sample part
	\begin{soln}
		content...
	\end{soln}
\end{assgnenum}
\end{qstn}
\end{document}
%	\begin{figure}[H]
%	\centering
%	\includegraphics[width=0.75\linewidth]{p}
%	\caption{caption.\label{fig:}}
%	\end{figure}